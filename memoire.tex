\documentclass[a4paper, 12pt]{book}
\usepackage{graphicx}
\usepackage[french]{babel}
\usepackage[utf8]{inputenc}
\usepackage{textcomp}
\usepackage[T1]{fontenc}
\usepackage{multirow}

\usepackage[table]{xcolor}

\usepackage{listings}
\usepackage{float}
\usepackage{url}
\usepackage[french]{algorithm}
\usepackage{style/myalgorithm}
\usepackage{amsmath,amsfonts,amssymb}
\usepackage{biblatex}
\addbibresource{memoire}
\newcommand{\fBm}{\emph{fBm}~}
\newcommand{\etal}{\emph{et al.}~}
\newcommand{\glAd}{\emph{GL4D}~}
\newcommand{\apiopengl}{API OpenGL\textsuperscript{\textregistered}~}
\newcommand{\opengl}{OpenGL\textsuperscript{\textregistered}~}
\newcommand{\opengles}{OpenGL\textsuperscript{\textregistered}ES~}
\newcommand{\clang}{langage \texttt{C}}
\newcommand{\codesource}{\textsc{Code source}~}
\floatstyle{ruled}
\newfloat{programslist}{htbp}{locs}
\newcommand{\listofprograms}{\listof{programslist}{Liste des codes source}}
\newcounter{program}[subsection]
\renewcommand{\theprogram}{\arabic{chapter}.\arabic{program}}

\newenvironment{program}[1]{
  \if\relax\detokenize{#1}\relax
  \gdef\mycaption{\relax}
  \else
  \gdef\mycaption{#1}
  \fi
  \refstepcounter{program}
  \addcontentsline{locs}{section}{#1}
  \footnotesize
}{
  \begin{description}
    \item[\codesource \theprogram]--~\mycaption
  \end{description}
}

\begin{document}
\begin{titlepage}
  \begin{center}
    %\begin{tabular*}{\textwidth}{l@{\extracolsep{\fill}}r}
      \includegraphics[height=2.5cm, width=6cm]{images/paris8Logo.png}
    %\end{tabular*}
    \small 
    \rule{\textwidth}{.5pt}~\\
    \large 
    \textsc{Université Paris 8 - Vincennes à Saint-Denis}\vspace{0.5cm}\\
    \textbf{Master 1 Informatique}\vspace{3.0cm}\\
    \Large
    \textbf{Mémoire projet tuteuré}\\
    \textbf{Qualification de caméras RGB-D}\vspace{1.5cm}\\
    
    \large
    \textbf{Yasmine BOUDJEMAÏ}\\
\textbf{Mélanie DE JESUS CORREIA}\vspace{1.5cm}\\
  \end{center}\vspace{3.5cm}~\\
  \begin{tabular}{ll}
    \hspace{-0.45cm}Organisme~:~&~Université Paris 8 Vincennes-Saint-Denis\\
    \hspace{-0.45cm}Tuteur~:~&~Farès  \textsc{BELHADJ}\\
  \end{tabular}
\end{titlepage}
\frontmatter



\chapter*{Dédicaces}
\markboth{\sc Dédicaces}{}


\chapter*{Remerciements}
\markboth{\sc Remerciements}{}

\chapter*{Résumé}
\markboth{\sc Résumé}{}


%% Table des matières
\tableofcontents
%% La liste des figure est optionnelle (si votre rapport manque de
%% contenu ajouter ce type de pages sera perçu négativement)
\listoffigures
%% La liste des programmes est optionnelle (si votre rapport manque de
%% contenu ajouter ce type de pages sera perçu négativement)
%\listofprogram
\mainmatter
\chapter*{Introduction}
\markboth{\sc Introduction}{}
\addcontentsline{toc}{chapter}{Introduction}
Introduction a faire à la fin.
\chapter{État de l'art}

Dans ce chapitre, nous allons définir ce qu'est une caméra RGB-D, nous énumérons certains des différents modèles existants, nous abordons brièvement les différentes techniques utilisées pour la récupération de la profondeur par une caméra. Enfin, nous discutons les différences recensées sur la \texttt{kinect V2} et \texttt{V3}.

\section{Description d'une caméra RGB-D}

La caméra RGB-D, aussi appelée capteur RGB-D, est une caméra fournissant en même temps une image couleur et une carte de profondeur caractérisant la distance des objets vus dans l'image. Cela est rendu possible grâce à un capteur RGB et un capteur de profondeur (D pour Depth). C'est principalement ces captures qui vont nous intéresser tout au lond des qualifications.


\section{Modèles existants}
Il existe différents modèles de caméras RGB-D. Parmi elles, nous pouvons citer la Kinect et ses différentes versions, Asus Xtion Pro Live, BlasterX Senz3D, Orbbec, Intel RealSens D415, ...
\par \texttt{La Kinect} a fait son apparition en septembre 2008. Elle a été conçue par Microsoft et était destinée pour la console de jeu XBox 360. Elle permettait aux utilisateurs d'interagir avec la console à l'aide d'une NUI \footnote{Natural User Interface (Interface Utilisateur Naturelle), se réfère à une interface utilisateur invisible.} en utilisant les mouvements gestuels et une reconnaissance vocale. Elle sera plus tard utilisée dans les domaines de la  recherche et du développement pour différents secteurs comme le domaine de la médecine, l'industrie automobile, la robotique, l'éducation,  .... 
\par \texttt{Asus Xtion Pro Live} est le modèle de référence que nous utilisons afin d'effectuer les qualifications. Elle utilise la technologie PrimeSense  \footnote{Connu principalement pour sa licence de conception matérielle et de puce employée dans le mécanisme de détection de mouvements de la Kinect XBox360. Pour plus d'informations, le lecteur peut se référer à ce lien \url{https://www.crunchbase.com/organization/primesense#section-web-traffic-by-similarweb}.} pour la détection de mouvements.
\par \texttt{BlasterX Senz3D} a fait son apparation en Septembre 2016. Conçue par Creative,  elle a été présentée comme une webcam intelligente basée sur ce qu'il y a de mieux en matière de savoir-faire et de technologie. Elle possède trois lentilles pour capturer les données visuelles: une caméra RVB, une caméra infra-rouge et un projecteur laser. Ces dernières collaborent avec la technologies Inetl RealSense pour réagir aux expressions faciales et aux gestes corporels des utilisateurs.
\par \texttt{Orbbec Astra Pro} fait partie de la série Astra. Elles offres une vision par ordinateur qui permet des dizaines de fonctions telles que la reconnaissance des visages, la reconnaissance des gestes, le suivi du corps humain, la mesure tridimensionnelle, la perception de l'environnement et la reconstruction de cartes en trois dimensions.  De plus, elles offres une réactivité haut de gamme, une mesure de la profondeur, des dégradés fluides et des contours précis, ainsi que la possibilité de filtrer les pixels de profondeur de faible qualité.
\par \texttt{Intel RealSense D415} a un champ de vision standard bien adapté aux applications de haute précision telles que la numérisation 3D. Elle comprend le processeur Intel RealSense Vision D4 offrant une résolution en profondeur élevée, des capacités de longue portée, une technologie d'obturation globale et un large champ de vision. Grâce à ces deux derniers, la caméra offre une perception précise de la profondeur lorsque l'objet est en mouvement ou que l'appareil est en marche. De plus, elle couvre un champ de vision plus large, minimisant ainsi les angles morts.

\section{Taux d'erreur (RMSE)}

\section{Théoriquement (RMSE)}

\chapter[Modèle logiciel pour la qualification]{Modèle logiciel pour la qualification de caméra RGB-D}
Dans ce passage, nous allons décrire notre outil conçu pour qualifier les caméras.

\section{Description de l'application}

\subsection{Première partie: Affichage de la scène captée }

\subsection{Seconde partie: Interface graphique}


\section{Conception du modèle réel et du modèle virtuel}

\section{Comparaison de la depth OpenGL avec la depth de la caméra RGB-D}
Dans ce passage, nous exposons certains des résultats obtenus ainsi que les résultats attendus avec la caméra Asus.

\subsection{Résultats obtenus}

\subsection{Résultats attendus}


\chapter{Cas pratiques de qualification}

\section{Modèle 3D}
\section{La POC}
\section{Résultats et critique}


\chapter{Conclusion et Perspectives\label{chap-conclusion}}


\nocite{*}
%	\bibliographystyle{alpha}
%\bibliography{memoire}
\printbibliography
\end{document}

